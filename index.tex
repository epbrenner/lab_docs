% Options for packages loaded elsewhere
\PassOptionsToPackage{unicode}{hyperref}
\PassOptionsToPackage{hyphens}{url}
\PassOptionsToPackage{dvipsnames,svgnames,x11names}{xcolor}
%
\documentclass[
  letterpaper,
  DIV=11,
  numbers=noendperiod]{scrreprt}

\usepackage{amsmath,amssymb}
\usepackage{lmodern}
\usepackage{iftex}
\ifPDFTeX
  \usepackage[T1]{fontenc}
  \usepackage[utf8]{inputenc}
  \usepackage{textcomp} % provide euro and other symbols
\else % if luatex or xetex
  \usepackage{unicode-math}
  \defaultfontfeatures{Scale=MatchLowercase}
  \defaultfontfeatures[\rmfamily]{Ligatures=TeX,Scale=1}
\fi
% Use upquote if available, for straight quotes in verbatim environments
\IfFileExists{upquote.sty}{\usepackage{upquote}}{}
\IfFileExists{microtype.sty}{% use microtype if available
  \usepackage[]{microtype}
  \UseMicrotypeSet[protrusion]{basicmath} % disable protrusion for tt fonts
}{}
\makeatletter
\@ifundefined{KOMAClassName}{% if non-KOMA class
  \IfFileExists{parskip.sty}{%
    \usepackage{parskip}
  }{% else
    \setlength{\parindent}{0pt}
    \setlength{\parskip}{6pt plus 2pt minus 1pt}}
}{% if KOMA class
  \KOMAoptions{parskip=half}}
\makeatother
\usepackage{xcolor}
\setlength{\emergencystretch}{3em} % prevent overfull lines
\setcounter{secnumdepth}{5}
% Make \paragraph and \subparagraph free-standing
\ifx\paragraph\undefined\else
  \let\oldparagraph\paragraph
  \renewcommand{\paragraph}[1]{\oldparagraph{#1}\mbox{}}
\fi
\ifx\subparagraph\undefined\else
  \let\oldsubparagraph\subparagraph
  \renewcommand{\subparagraph}[1]{\oldsubparagraph{#1}\mbox{}}
\fi


\providecommand{\tightlist}{%
  \setlength{\itemsep}{0pt}\setlength{\parskip}{0pt}}\usepackage{longtable,booktabs,array}
\usepackage{calc} % for calculating minipage widths
% Correct order of tables after \paragraph or \subparagraph
\usepackage{etoolbox}
\makeatletter
\patchcmd\longtable{\par}{\if@noskipsec\mbox{}\fi\par}{}{}
\makeatother
% Allow footnotes in longtable head/foot
\IfFileExists{footnotehyper.sty}{\usepackage{footnotehyper}}{\usepackage{footnote}}
\makesavenoteenv{longtable}
\usepackage{graphicx}
\makeatletter
\def\maxwidth{\ifdim\Gin@nat@width>\linewidth\linewidth\else\Gin@nat@width\fi}
\def\maxheight{\ifdim\Gin@nat@height>\textheight\textheight\else\Gin@nat@height\fi}
\makeatother
% Scale images if necessary, so that they will not overflow the page
% margins by default, and it is still possible to overwrite the defaults
% using explicit options in \includegraphics[width, height, ...]{}
\setkeys{Gin}{width=\maxwidth,height=\maxheight,keepaspectratio}
% Set default figure placement to htbp
\makeatletter
\def\fps@figure{htbp}
\makeatother
\newlength{\cslhangindent}
\setlength{\cslhangindent}{1.5em}
\newlength{\csllabelwidth}
\setlength{\csllabelwidth}{3em}
\newlength{\cslentryspacingunit} % times entry-spacing
\setlength{\cslentryspacingunit}{\parskip}
\newenvironment{CSLReferences}[2] % #1 hanging-ident, #2 entry spacing
 {% don't indent paragraphs
  \setlength{\parindent}{0pt}
  % turn on hanging indent if param 1 is 1
  \ifodd #1
  \let\oldpar\par
  \def\par{\hangindent=\cslhangindent\oldpar}
  \fi
  % set entry spacing
  \setlength{\parskip}{#2\cslentryspacingunit}
 }%
 {}
\usepackage{calc}
\newcommand{\CSLBlock}[1]{#1\hfill\break}
\newcommand{\CSLLeftMargin}[1]{\parbox[t]{\csllabelwidth}{#1}}
\newcommand{\CSLRightInline}[1]{\parbox[t]{\linewidth - \csllabelwidth}{#1}\break}
\newcommand{\CSLIndent}[1]{\hspace{\cslhangindent}#1}

\KOMAoption{captions}{tableheading}
\makeatletter
\makeatother
\makeatletter
\@ifpackageloaded{bookmark}{}{\usepackage{bookmark}}
\makeatother
\makeatletter
\@ifpackageloaded{caption}{}{\usepackage{caption}}
\AtBeginDocument{%
\ifdefined\contentsname
  \renewcommand*\contentsname{Table of contents}
\else
  \newcommand\contentsname{Table of contents}
\fi
\ifdefined\listfigurename
  \renewcommand*\listfigurename{List of Figures}
\else
  \newcommand\listfigurename{List of Figures}
\fi
\ifdefined\listtablename
  \renewcommand*\listtablename{List of Tables}
\else
  \newcommand\listtablename{List of Tables}
\fi
\ifdefined\figurename
  \renewcommand*\figurename{Figure}
\else
  \newcommand\figurename{Figure}
\fi
\ifdefined\tablename
  \renewcommand*\tablename{Table}
\else
  \newcommand\tablename{Table}
\fi
}
\@ifpackageloaded{float}{}{\usepackage{float}}
\floatstyle{ruled}
\@ifundefined{c@chapter}{\newfloat{codelisting}{h}{lop}}{\newfloat{codelisting}{h}{lop}[chapter]}
\floatname{codelisting}{Listing}
\newcommand*\listoflistings{\listof{codelisting}{List of Listings}}
\makeatother
\makeatletter
\@ifpackageloaded{caption}{}{\usepackage{caption}}
\@ifpackageloaded{subcaption}{}{\usepackage{subcaption}}
\makeatother
\makeatletter
\@ifpackageloaded{tcolorbox}{}{\usepackage[many]{tcolorbox}}
\makeatother
\makeatletter
\@ifundefined{shadecolor}{\definecolor{shadecolor}{rgb}{.97, .97, .97}}
\makeatother
\makeatletter
\makeatother
\ifLuaTeX
  \usepackage{selnolig}  % disable illegal ligatures
\fi
\IfFileExists{bookmark.sty}{\usepackage{bookmark}}{\usepackage{hyperref}}
\IfFileExists{xurl.sty}{\usepackage{xurl}}{} % add URL line breaks if available
\urlstyle{same} % disable monospaced font for URLs
\hypersetup{
  pdftitle={Lab setup},
  pdfauthor={Janani Ravi},
  colorlinks=true,
  linkcolor={blue},
  filecolor={Maroon},
  citecolor={Blue},
  urlcolor={Blue},
  pdfcreator={LaTeX via pandoc}}

\title{Lab setup}
\usepackage{etoolbox}
\makeatletter
\providecommand{\subtitle}[1]{% add subtitle to \maketitle
  \apptocmd{\@title}{\par {\large #1 \par}}{}{}
}
\makeatother
\subtitle{JRaviLab}
\author{Janani Ravi}
\date{}

\begin{document}
\maketitle
\ifdefined\Shaded\renewenvironment{Shaded}{\begin{tcolorbox}[enhanced, borderline west={3pt}{0pt}{shadecolor}, interior hidden, breakable, sharp corners, boxrule=0pt, frame hidden]}{\end{tcolorbox}}\fi

\renewcommand*\contentsname{Table of contents}
{
\hypersetup{linkcolor=}
\setcounter{tocdepth}{2}
\tableofcontents
}
\bookmarksetup{startatroot}

\hypertarget{jravilab}{%
\chapter*{JRaviLab}\label{jravilab}}
\addcontentsline{toc}{chapter}{JRaviLab}

This is a short book (built with
\href{https://quarto.org/docs/books}{Quarto}) compiled to guide,
onboard, and offboard JRaviLab members at the University of Colorado
Anschutz Medical Campus.

\hypertarget{acknowledgments}{%
\subsection*{Acknowledgments}\label{acknowledgments}}
\addcontentsline{toc}{subsection}{Acknowledgments}

We appreciate and build upon awesome onboarding group resources from
Arjun Krishnan, Fan Zhang, Annika Barber, Christoph Rau, and several
others.

\bookmarksetup{startatroot}

\hypertarget{onboarding}{%
\chapter{Onboarding}\label{onboarding}}

Welcome to JRaviLab! We are excited that you are here --- as a student,
postdoc, or a visiting researcher!

\bookmarksetup{startatroot}

\hypertarget{slack}{%
\chapter{Slack}\label{slack}}

First and foremost, let's get you on our
\href{jravilab.slack.com}{Slack}. All our conversations \& every last
bit of science gets done here! :) So, please install this on your
desktops (phones) and turn on notifications during working hours,
whenever that is!

\hypertarget{getting-started}{%
\section{Getting started}\label{getting-started}}

When you join, please introduce yourself in \texttt{\#general}. Tell us
a little about yourself here --- where you are from, what are your
interests -- science-wise or otherwise, why are you interested in
working with us, and in what capacity you'll be joining us.

I've never used Slack -\/- where do I even start? Maybe you can start
with the
\href{https://github.com/JRaviLab/cheatsheets/blob/master/slack-misc/slack-cheatsheet.pdf}{Slack
cheatsheet} and
\href{https://slack.com/help/articles/201374536-Slack-keyboard-shortcuts}{Keyboard
shortcuts}.

\hypertarget{what-next}{%
\section{What next?}\label{what-next}}

Next, you can join the different channels to participate in various
kinds of conversations with the group --- look at the channel
description, check out the pinned messages of the channel, and dive
right in! Here are a few examples.

Finally, you will be invited to specific project channels (based on the
primary and secondary projects you will be working on). Based on chats
with your colleagues, if you'd like to contribute to/give feedback to
other projects, feel free to join those channels too.

\hypertarget{tips}{%
\section{💡Tips}\label{tips}}

\hypertarget{learning-r}{%
\subsection{Learning R}\label{learning-r}}

If you are new to R programming, join the \texttt{\#bootcamp} channel.
Use the \texttt{\#howto-x} channels to ask for help/answer others'
questions on R/Py programming, shell scripting, version controlling, or
anything else that's general and not project-specific.

Check out the Slack tips in the pinned posts in
\href{https://jravilab.slack.com/archives/CATMCKFT9}{\#how-to-git} to
get oriented since you are new to Slack. The channel used to be
\#how-to-slack!

To learn R, I would recommend getting started with the pinned posts in
\href{https://jravilab.slack.com/archives/CARJ72W3U}{\#courses-primers}.\\
A couple of helpful GitHub repositories

\begin{itemize}
\item
  \url{https://github.com/jananiravi/compbio-gists}
  (R/Python/Unix/vi/GitHub)
\item
  \url{https://github.com/jananiravi/workshop-tidyverse} (Intro to
  \texttt{R\textquotesingle{}s\ tidyverse\ package} --- very useful to
  learn right away)
\item
  Other R workshop materials from our R-Ladies East Lansing chapter:
  \url{https://github.com/rladies-eastlansing}
\item
  Interactive tutorials with
  \href{https://rstudio.github.io/learnr/}{learnr} \&
  \href{https://swirlstats.com/}{swirl}.
\item
  \href{https://datacarpentry.org/R-genomics/index.html}{R for Genomics
  from Data Carpentry}
\item
  Coursera

  \begin{itemize}
  \item
    \href{https://www.coursera.org/learn/r-programming-tidyverse}{Intro
    to R programming and tidyverse}
  \item
    \href{https://www.coursera.org/specializations/jhu-data-visualization-dashboarding-with-r}{Data
    visualization and Dashboarding with R specialization}
  \item
    \href{https://www.coursera.org/learn/jhu-getting-started-data-viz-r}{Getting
    Started with Data Visualization in R}
  \item
    \ldots{} and more \textbar{} explore by topic, duration, skill-level
  \end{itemize}
\end{itemize}

📌 Go-to books (also in pinned posts on Slack)

\begin{itemize}
\item
  \href{https://r4ds.had.co.nz/}{R for Data Science} (for tidyverse and
  such)
\item
  \href{https://rstudio-education.github.io/hopr/}{Hands-On Programming
  with R} (for base R)
\end{itemize}

\bookmarksetup{startatroot}

\hypertarget{github}{%
\chapter{GitHub}\label{github}}

If you haven't already, please create a professional GitHub account
(\emph{e.g.,} \texttt{jananiravi}). Once you pass that along to us, we
will you add you to the \href{//github.com/jravilab}{JRaviLab GitHub
organization}. Also, a neat memorable username will give you the
opportunity to host your own webpage (\emph{e.g.,}
\texttt{jravilab.github.io} or \texttt{jananiravi.github.io}).

Intro to Git, GitHub resources

\begin{itemize}
\item
  \url{https://happygitwithr.com/} Happy Git and GitHub for the useR
  (connecting git/GitHub w/ R)
\item
  \href{https://docs.github.com/en/get-started/quickstart/git-and-github-learning-resources}{Git
  and GitHub learning resources} from GitHub
\item
  \href{https://product.hubspot.com/blog/git-and-github-tutorial-for-beginners}{Git
  101}
\item
  \href{https://www.coursera.org/learn/introduction-git-github}{Coursera
  Intro to Git and GitHub course}
\end{itemize}

\hypertarget{vpn}{%
\section{VPN}\label{vpn}}

https://www.ucdenver.edu/offices/office-of-information-technology/software/how-do-i-use/vpn-and-remote-access

\bookmarksetup{startatroot}

\hypertarget{meetings}{%
\chapter{Meetings}\label{meetings}}

Associated repo: \textless github.com/jravilab/group\textgreater{}

\bookmarksetup{startatroot}

\hypertarget{funding-resources}{%
\chapter{Funding Resources}\label{funding-resources}}

We strongly encourage trainees to apply for scholarships/fellowships or
other grants to help support their independent research and to get
acquainted with the process of developing competitive research and
personal statements.

Here are a few scholarship and funding opportunities:

\hypertarget{grad-students}{%
\section{Grad students}\label{grad-students}}

\hypertarget{cu-funding}{%
\subsection{CU funding}\label{cu-funding}}

\url{https://gs.ucdenver.edu/funding/} Graduate tuition for in-state,
out of state / non-resident / internationals:
\url{https://www.cuanschutz.edu/student-finances/tuition-fees/graduate}

Immigrants vs resident aliens:
\url{https://highered.colorado.gov/immigrant-or-resident-alien}

\hypertarget{training-grants}{%
\subsection{Training grants}\label{training-grants}}

\url{https://www.cuanschutz.edu/graduate-programs/biomedical-sciences-program/resources/grants-and-fellowships}

\hypertarget{awis}{%
\subsection{AWIS}\label{awis}}

\url{https://awis.memberclicks.net/}

\hypertarget{undergrads}{%
\section{Undergrads}\label{undergrads}}

\hypertarget{colorado-biomedical-informatics-summer-training-fellowship-for-urm}{%
\subsection{Colorado Biomedical Informatics Summer Training Fellowship
for
URM}\label{colorado-biomedical-informatics-summer-training-fellowship-for-urm}}

\url{https://www.cuanschutz.edu/docs/librariesprovider233/compbioscience/deadline-extended_sttp-2022-flyer.pdf}.
Mail Caitlyn Moloney with questions:
\href{mailto:caitlin.moloney@cuanschutz.edu}{\nolinkurl{caitlin.moloney@cuanschutz.edu}}

\hypertarget{pike-prep}{%
\subsection{PIKE PREP}\label{pike-prep}}

University of Colorado (CU) Anschutz Medical Campus Preparation in
Interdisciplinary Knowledge to Excel (PIKE) PREP:
\url{https://medschool.cuanschutz.edu/colorado-cancer-center/education/pike-prep}
(URM + citizen + BS in biomedical field)

For high school, undergraduate, and graduate/professional students:
\url{https://graduateschool.cuanschutz.edu/programs-of-study/summer-programs}

\hypertarget{postdocs}{%
\section{Postdocs}\label{postdocs}}

Beautiful central resource maintained by JHU:
\url{https://research.jhu.edu/rdt/funding-opportunities/postdoctoral/}

\hypertarget{visiting-researchers}{%
\section{Visiting researchers}\label{visiting-researchers}}

\bookmarksetup{startatroot}

\hypertarget{other-cu-resources}{%
\chapter{Other CU resources}\label{other-cu-resources}}

\hypertarget{women-in-stem}{%
\section{Women in STEM}\label{women-in-stem}}

\url{https://www.cuanschutz.edu/services/women-in-stem/resources}

\hypertarget{central-office-of-diversity-equity-inclusion-and-community-engagement}{%
\section{Central Office of Diversity, Equity, Inclusion and Community
Engagement}\label{central-office-of-diversity-equity-inclusion-and-community-engagement}}

\url{https://www.cuanschutz.edu/offices/diversity-equity-inclusion-community}

\hypertarget{office-of-disability-access-and-inclusion}{%
\subsection{Office of Disability, Access, and
Inclusion}\label{office-of-disability-access-and-inclusion}}

\url{https://www.cuanschutz.edu/offices/office-of-disability-access-and-inclusion/home-page}

\hypertarget{office-of-diversity-and-inclusion}{%
\subsection{Office of Diversity and
Inclusion}\label{office-of-diversity-and-inclusion}}

\url{https://medschool.cuanschutz.edu/deans-office/diversity-inclusion}

\hypertarget{writing-center-at-cu-anschutz}{%
\section{Writing Center at CU
Anschutz}\label{writing-center-at-cu-anschutz}}

\url{https://clas.ucdenver.edu/writing-center/locations/writing-center-cu-anschutz}

\bookmarksetup{startatroot}

\hypertarget{offboarding}{%
\chapter{Offboarding}\label{offboarding}}

We are delighted to have been a part of your research life. We wish you
great laurels as you move on to the next phase of your career.

Please make sure you have checked these boxes before leaving our group.

\begin{itemize}
\item[$\square$]
  All your code is on GitHub
\item[$\square$]
  All your processed data files/figures/reports (\textless100mb) are on
  GitHub
\item[$\square$]
  Large data files (raw/processed) are on the server (and backed up)
\item[$\square$]
  All your scripts and data files are well-annotated with appropriate
  README files
\item[$\square$]
  Added detailed status report of where things -- what was tried, what
  worked/didn't work, where the scripts/data files are, what remains to
  be done (within the scope of the project, and clear next steps)
\item[$\square$]
  You have reoriented me and at least 1-2 other project members with the
  precise status (and next steps) of the project and location of files
\item[$\square$]
  You are not a full member of the GitHub organization or Slack anymore
  but will retain access to the project channel/repo until publication.
\end{itemize}

\bookmarksetup{startatroot}

\hypertarget{references}{%
\chapter*{References}\label{references}}
\addcontentsline{toc}{chapter}{References}

\hypertarget{refs}{}
\begin{CSLReferences}{0}{0}
\end{CSLReferences}



\end{document}
